% resume.tex
% vim:set ft=tex spell:

\documentclass[10pt,letterpaper]{article}
\usepackage[letterpaper,margin=0.75in]{geometry}
\usepackage[utf8]{inputenc}
\usepackage{mdwlist}
\usepackage[T1]{fontenc}
\usepackage{textcomp}
\usepackage{tgpagella}
\usepackage{hyperref}
\pagestyle{empty}
\setlength{\tabcolsep}{0em}

% indentsection style, used for sections that aren't already in lists
% that need indentation to the level of all text in the document
\newenvironment{indentsection}[1]%
{\begin{list}{}%
	{\setlength{\leftmargin}{#1}}%
	\item[]%
}
{\end{list}}

% opposite of above; bump a section back toward the left margin
\newenvironment{unindentsection}[1]%
{\begin{list}{}%
	{\setlength{\leftmargin}{-0.5#1}}%
	\item[]%
}
{\end{list}}

% format two pieces of text, one left aligned and one right aligned
\newcommand{\headerrow}[2]
{\begin{tabular*}{\linewidth}{l@{\extracolsep{\fill}}r}
	#1 &
	#2 \\
\end{tabular*}}

% make "C++" look pretty when used in text by touching up the plus signs
\newcommand{\CPP}
{C\nolinebreak[4]\hspace{-.05em}\raisebox{.22ex}{\footnotesize\bf ++}}

% and the actual content starts here
\begin{document}

\begin{center}
{\LARGE \textbf{Jonathan D. Fok kan}}

4885 Blvd Henri-Bourassa O.\ \ \textbullet
\ \ Apt\ 323\ \ \textbullet
\ \ Montr\'{e}al, QC H4L 0A5
\\
(514) 963-2699\ \ \textbullet
\ \ jfokkan@gmail.com
\end{center}

% Double-line for start and end of epigraph.
%\newcommand{\epiline}{\hrule \vskip -.2em \hrule}

\hrule \vskip -.2em \hrule

%\hrule

\vspace{-0.4em}
\subsection*{SKILLS}

%\subsubsection*{Languages}
%\begin{indentsection}{\parindent}
%\hyphenpenalty=1000
\begin{description*}
	\item[Languages:]
        French, English and Cr\'{e}ole
\end{description*}
%\end{indentsection}

\subsubsection*{Core Technical \& Creative Skills}
\begin{indentsection}{\parindent}
\hyphenpenalty=1000
\begin{description*}
	\item[Languages:]
	Scala, Go, Java, C\#, Python, Javascript, shell, HTML/CSS, SQL, C, \LaTeX, SML
	\item[Tools:]
        Git, IntelliJ, Emacs, Eclipse, Vim
        \item[Paradigms:]
        Object Oriented, Functional, Declarative, Imperative, Functional Reactive (FRP)
        \item[Design:]
        UML, ER Model
\end{description*}
\end{indentsection}


\hrule
\vspace{-0.4em}
\subsection*{EDUCATION}

\begin{itemize}
	\parskip=0.1em

	\item
	\headerrow
		{\textbf{McGill University}}
		{\textbf{Montr\'{e}al, QC}}
	\\
	\headerrow
		{\emph{B.Sc Hon Software Engineering}}
		{\emph{January 2011 -- 2014}}
%	\begin{itemize*}
%		\item Lorem ipsum dolor sit amet, consectetuer adipiscing elit.
%		\item Mirum est notare quam littera gothica, quam nunc putamus parum
%		claram.
%	\end{itemize*}

	\item
	\headerrow
		{\textbf{C\'{e}gep Vanier College}}
		{\textbf{Montr\'{e}al, QC}}
	\\
	\headerrow
		{\emph{DEC, Pure and Applied Science}}
		{\emph{January 2009 -- December 2010}}
	\begin{itemize*}
		\item \textbf{Honour Roll} for Pure and Applied Science
	\end{itemize*}

	\item
	\headerrow
		{\textbf{Saint Mary's College}}
		{\textbf{\^{i}le Maurice}}
	\\
	\headerrow
		{\emph{Cambridge, GCE O-Levels, Science}}
		{\emph{January 2004 -- November 2008}}
\end{itemize}


\hrule
\vspace{-0.4em}
\subsection*{SELECTED PROJECTS}

\begin{itemize}
	\parskip=0.1em

	\item
	\headerrow
		{\textbf{SSMU Powerlifting Team Website}}
		{\emph{March 2015}}
	\\
	\headerrow
		{\emph{A team website for the SSMU powerlifting team at McGill to show team members'
                    the progress, competition results and records}}
		{\emph{}}
	\begin{itemize*}
		\item Scala project with the Play framework
		\item Backend in PostgreSQL with the Slick Functional Relational Mapping library
                \item Front-end made using Bootstrap and JQuery
                \item Repository at \url{https://github.com/jonfk/ssmu-powerlifting-play}
	\end{itemize*}
        % End Project

	\item
	\headerrow
		{\textbf{Opendaylight Project Explorer}}
		{\emph{January 2015}}
	\\
	\headerrow
		{\emph{An xml and Maven Repository file crawler for the Opendaylight Project}}
		{\emph{}}
	\begin{itemize*}
		\item Scala project with the AKKA framework
		\item Backend in Spray with eclipse aether library to crawl the Maven Repository
                \item Front-end made using Bootstrap, JQuery, AngularJS and D3.js
                \item A root xml is crawled and parsed which points to xml files that are recursively
                  parsed to show the contents and relationships between Opendaylight projects in
                  terms of Karaf features
	\end{itemize*}
        % End Project

	\item
	\headerrow
		{\textbf{Dynamic Functional Programming language}}
		{\emph{2015 - Work In Progress}}
	\\
	\headerrow
		{\emph{A purely functional dynamic programming language with ML like syntax in Go}}
		{\emph{}}
	\begin{itemize*}
		\item Lexer and parser are hand written in Go
                \item Concurrent lexer and recursive descent parser
                \item Work in Progress with only expressions currently working
                \item Repository at \url{https://github.com/jonfk/calc}
	\end{itemize*}
        % End Project

	\item
	\headerrow
		{\textbf{McGill Robotics Simulation Team}}
		{\emph{Winter 2013}}
	\\
	\headerrow
		{\emph{A simulation environment to test computer vision, path planning and control of an underwater robot}}
		{\emph{}}
	\begin{itemize*}
		\item Team Project of 2 people
		\item Simulation environment was built using Gazebo and ROS in C\texttt{++}
		\item Simulate approximate behaviour of objects without fluid dynamics
		\item Interface simulation with control systems and path planning
	\end{itemize*}
        % End Project

	\item
	\headerrow
		{\textbf{Honours Thesis: Fault Tolerance in Software Defined Networks}}
		{\emph{Winter 2013}}
	\\
	\headerrow
		{\emph{Implement and test a method of replicating OpenFlow Controllers in a Software Defined Network}}
		{\emph{}}
	\begin{itemize*}
		\item Used Mininet as a testing environment
		\item Modified OpenFlow switches using python to implement a polling mechanism to test which controller to use
                \item Implemented a fallback algorithm for OpenFlow switches to improve fault tolerance
	\end{itemize*}
        % End Project

	\item
	\headerrow
		{\textbf{Android Directional application for the visually impaired}}
		{\emph{Winter 2013}}
	\\
	\headerrow
		{\emph{An application that provides audio cues and directions to navigate urban environments}}
		{\emph{}}
	\begin{itemize*}
		\item Used android talk back features to provide a usable interface for the visually impaired
		\item Designed in a User Centered Design approach focussed on iterative feedback from users
		\item Developped while testing with the help of several visually impaired users
                \item App developped using Java
	\end{itemize*}
        % End Project

\end{itemize}


\hrule
\vspace{-0.4em}
\subsection*{WORK EXPERIENCE}

\begin{itemize}
	\parskip=0.1em

	\item
	\headerrow
		{\textbf{OMSignal Inc.}}
		{\textbf{Montr\'{e}al, QC}}
	\\
	\headerrow
		{\emph{Full Stack Software Developer}}
		{\emph{March 2015 -- Current}}
	\begin{description*}
                \item[Technologies used:] Scala/Akka, ObjectiveC/Swift, Go, Javascript/React, Docker
	\end{description*}
	\begin{itemize*}
                \item OMsignal is a wearable tech company with an ios app that tracks the user's biometric signals
                \item Led the implementation of Key Performance Indicators based on the data from our backend to gain insight into how users were reacting to our product
                \item Devised and implemented interconnection from the ios app and backend to third party services such as Strava, Nike+ and UnderArmour
	\end{itemize*}
        % End of Job

	\item
	\headerrow
		{\textbf{Inocybe Technologies Inc.}}
		{\textbf{Montr\'{e}al, QC}}
	\\
	\headerrow
		{\emph{Software Developer}}
		{\emph{May 2014 -- March 2015}}
	\begin{description*}
                \item[Technologies used:] Docker, Go, Scala/Akka/Play framework, Java/OSGi, Javascript/AngularJS
	\end{description*}
	\begin{itemize*}
		\item Customized and built CoreOS (a minimal linux distribution) to run as an SDN controller appliance
                \item Implemented an update manager in Go based on Google's Omaha protocol for our integrated software product
                \item Designed and implemented a Continuous Integration pipeline
	          \begin{itemize*}
                    \item Implemented using GoCD with a small cluster of workers
                    \item Using Docker for isolating build environments
                    \item CI pipeline ran tests, build code and pushed build artifacts to Artifactory repository
	          \end{itemize*}
                \item Created a project explorer for the Opendaylight project showing the karaf features and dependencies of modules in a complex project
	\end{itemize*}
        % End of Job

	\item
	\headerrow
		{\textbf{Business Process Management Education Network}}
		{\textbf{University of Coimbra, Portugal}}
	\\
	\headerrow
		{\emph{Research Intern}}
		{\emph{Summer 2012}}
	\begin{itemize*}
		\item Research project investigating the performance of at the time popular distributed databases for Data Warehousing use cases
		\item Wrote benchmarks to compare hadoop+hive, MySQL cluster and VoltDB
		\item Extended an inhouse framework for a distributed database based on MySQL
		\item Set up and maintained a cluster of about 10 linux machines for our work
	\end{itemize*}
        % End of Job

\end{itemize}

\hrule
\vspace{-0.4em}
\subsection*{ACTIVITIES \& INTERESTS}

\begin{indentsection}{\parindent}
\hyphenpenalty=1000
\begin{description*}
        \item[]Member of IEEE, McGill Branch (2011 -- Present)
        \item[]Powerlifting (Member of Qu\'{e}bec Powerlifting Federation) \& Cycling
        \item[]Photography
\end{description*}
\end{indentsection}

\hrule
\vspace{-0.4em}

\end{document}
